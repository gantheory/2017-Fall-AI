% Preamble {{{
% 紙張大小設定% {{{
\documentclass[12pt, a4paper]{article}
% \paperwidth=65cm
% \paperheight=160cm
% }}}
% usepackage {{{
\usepackage[margin=3cm]{geometry} % 上下左右距離邊緣2cm
\usepackage{mathtools,amsthm,amssymb} % 引入 AMS 數學環境
\usepackage{yhmath}      % math symbol
\usepackage{bm}      % bold math symbol
\usepackage{graphicx}    % 圖形插入用
\usepackage{fontspec}    % 加這個就可以設定字體
\usepackage{type1cm}    % 設定fontsize用
\usepackage{titlesec}   % 設定section等的字體
\usepackage{titling}    % 加強 title 功能
\usepackage{fancyhdr}   % 頁首頁尾
\usepackage{tabularx}   % 加強版 table
\usepackage[square, comma, numbers, super, sort&compress]{natbib}
% cite加強版
\usepackage[unicode, pdfborder={0 0 0}, bookmarksdepth=-1]{hyperref}
% ref加強版
\usepackage[usenames, dvipsnames]{color}  % 可以使用顏色
\usepackage[shortlabels, inline]{enumitem}  % 加強版enumerate
\usepackage{xpatch}

% pseudo code
\usepackage{algorithm}
\usepackage{algpseudocode}
\usepackage{amsmath}
\usepackage{graphics}
\usepackage{epsfig}

\graphicspath{{images/}}
% \usepackage{tabto}      % tab
% \usepackage{soul}       % highlight
% \usepackage{ulem}       % 字加裝飾
% \usepackage{wrapfig}     % 文繞圖
% \usepackage{floatflt}    % 浮動 figure
% \usepackage{float}       % 浮動環境
% \usepackage{caption}    % caption 增強
% \usepackage{subcaption}    % subfigures
% \usepackage{setspace}    % 控制空行
% \usepackage{mdframed}   % 可以加文字方框
% \usepackage{multicol}   % 多欄
% \usepackage[abbreviations]{siunitx} % SI unit
% \usepackage{dsfont}     % more mathbb
% }}}
% Tikz {{{
% \usepackage{tikz}
% \usepackage{circuitikz}
% }}}
% chinese environment {{{
\usepackage[CheckSingle, CJKmath]{xeCJK}  % xelatex 中文
\usepackage{CJKulem} % 中文字裝飾
\setCJKmainfont{Noto Sans CJK TC}
% 設定中文為系統上的字型,而英文不去更動,使用原TeX字型

% \XeTeXlinebreaklocale "zh"             %這兩行一定要加,中文才能自動換行
% \XeTeXlinebreakskip = 0pt plus 1pt     %這兩行一定要加,中文才能自動換行
% }}}
% 頁面設定 {{{
\newcolumntype{C}[1]{>{\centering\arraybackslash}p{#1}}
\setlength{\headheight}{15pt}  %with titling
\setlength{\droptitle}{-1.5cm} %title 與上緣的間距
% \posttitle{\par\end{center}} % title 與內文的間距
\parindent=24pt %設定縮排的距離
% \parskip=1ex  %設定行距
% \pagestyle{empty}  % empty: 無頁碼
% \pagestyle{fancy}  % fancy: fancyhdr

% use with fancygdr
% \lhead{\leftmark}
% \chead{}
% \rhead{}
% \lfoot{}
% \cfoot{}
% \rfoot{\thepage}
% \renewcommand{\headrulewidth}{0.4pt}
% \renewcommand{\footrulewidth}{0.4pt}

% \fancypagestyle{firststyle}
% {
  % \fancyhf{}
  % \fancyfoot[C]{\footnotesize Page \thepage\ of \pageref{LastPage}}
  % \renewcommand{\headrule}{\rule{\textwidth}{\headrulewidth}}
% }
% }}}
% insert code {{{
\usepackage{listings}
\usepackage{color}

\definecolor{dkgreen}{rgb}{0,0.6,0}
\definecolor{gray}{rgb}{0.5,0.5,0.5}
\definecolor{mauve}{rgb}{0.58,0,0.82}

\lstset{
    frame=tb,
    language=Python,
    aboveskip=3mm,
    belowskip=3mm,
    showstringspaces=false,
    columns=flexible,
    basicstyle={\small\ttfamily},
    numbers=none,
    numberstyle=\tiny\color{gray},
    keywordstyle=\color{blue},
    commentstyle=\color{dkgreen},
    stringstyle=\color{mauve},
    breaklines=true,
    breakatwhitespace=true,
    tabsize=3
}
% Load from file
% \lstinputlisting[breaklines]{filename}
% }}}
% change the number of max columns in matrix {{{
\setcounter{MaxMatrixCols}{20}
% }}}
% }}}

%\title{}
%\author{}
%\date{\today}

\begin{document}
%\maketitle
{\bf \noindent
\rule[3pt]{\textwidth}{0.3pt}\\
2017 Fall AI\hfill National Taiwan University \\
Instructor: Tian-Li Yu\hfill B03901133 Gary Shih \\
HW2 \hfill \today\\
\vspace{-10pt} \\
\rule[3pt]{\textwidth}{1.3pt}\\
[-1cm]
}

\section*{Question 5}
In question 5, I implement the following techniques:
\begin{enumerate}
  \item Real Distance \\
    Considering walls and reasonable ghosts, we can more reliably calculate distances to all points
    than Manhattan Distance by Breadth-First Search. Then, once we have distances to foods, capsules
    and ghosts, we will be able to let Pacman know where it should go in some priorities. \par
    However, how do we define reasonble ghosts? We know that there are two states of ghost, normal
    and scared. We should avoid to meet normal ghosts while in the case of scared ghosts, if we can
    catch them before they become normal, we can get much more score. Therefore, the definition of a
    reasonable ghost in my design is that a normal ghost or a scared ghost which Pacman approximately
    is not able to catch it before it becomes normal.
  \item Rewards \\
    After we know distances to those useful targets, we can apply some value to them and let Pacman
    know where it should go and avoid. In my implementation, I do linear combination to the
    reciprocal of these distances and the following is the weights I use:
    \begin{itemize}
      \item Food: 1.0
      \item Capsule: 100.0
      \item Scared ghost Pacman can catch: 200.0
    \end{itemize}
  \item currentGameState.getScore() \\
    This part is the same as scoreEvaluationFunction which provide scores that includes basic win,
    lose, and time penalty conditions.
\end{enumerate}

\section*{Note}
  An important point is that I do not additionally penalize the condition where Pacman may meet a
  normal ghost because I found that if I do, Pacman will become too conservative to eat foods and
  capsules. Besides, in our testcase, there are just two ghosts randomly move. As a result,
  currentGameState.getScore() is enough to let Pacman avoid ghosts.

\end{document}
% for default folding
% vim:fdm=marker:foldlevel=0
