%! TEX program = xelatex
% Preamble {{{
% 紙張大小設定% {{{
\documentclass[12pt, a4paper]{article}
% \paperwidth=65cm
% \paperheight=160cm
% }}}
% usepackage {{{
\usepackage[margin=3cm]{geometry} % 上下左右距離邊緣2cm
\usepackage{mathtools,amsthm,amssymb} % 引入 AMS 數學環境
\usepackage{yhmath}      % math symbol
\usepackage{bm}      % bold math symbol
\usepackage{graphicx}    % 圖形插入用
\usepackage{fontspec}    % 加這個就可以設定字體
\usepackage{type1cm}    % 設定fontsize用
\usepackage{titlesec}   % 設定section等的字體
\usepackage{titling}    % 加強 title 功能
\usepackage{fancyhdr}   % 頁首頁尾
\usepackage{tabularx}   % 加強版 table
\usepackage[square, comma, numbers, super, sort&compress]{natbib}
% cite加強版
\usepackage[unicode, pdfborder={0 0 0}, bookmarksdepth=-1]{hyperref}
% ref加強版
\usepackage[usenames, dvipsnames]{color}  % 可以使用顏色
\usepackage[shortlabels, inline]{enumitem}  % 加強版enumerate
\usepackage{xpatch}

% pseudo code
\usepackage{algorithm}
\usepackage{algpseudocode}
\usepackage{amsmath}
\usepackage{graphics}
\usepackage{epsfig}

\graphicspath{{images/}}
% \usepackage{tabto}      % tab
% \usepackage{soul}       % highlight
% \usepackage{ulem}       % 字加裝飾
% \usepackage{wrapfig}     % 文繞圖
% \usepackage{floatflt}    % 浮動 figure
% \usepackage{float}       % 浮動環境
% \usepackage{caption}    % caption 增強
% \usepackage{subcaption}    % subfigures
% \usepackage{setspace}    % 控制空行
% \usepackage{mdframed}   % 可以加文字方框
% \usepackage{multicol}   % 多欄
% \usepackage[abbreviations]{siunitx} % SI unit
% \usepackage{dsfont}     % more mathbb
% }}}
% Tikz {{{
% \usepackage{tikz}
% \usepackage{circuitikz}
% }}}
% chinese environment {{{
\usepackage[CheckSingle, CJKmath]{xeCJK}  % xelatex 中文
\usepackage{CJKulem} % 中文字裝飾
\setCJKmainfont{Noto Sans CJK TC}
% 設定中文為系統上的字型,而英文不去更動,使用原TeX字型

% \XeTeXlinebreaklocale "zh"             %這兩行一定要加,中文才能自動換行
% \XeTeXlinebreakskip = 0pt plus 1pt     %這兩行一定要加,中文才能自動換行
% }}}
% 頁面設定 {{{
\newcolumntype{C}[1]{>{\centering\arraybackslash}p{#1}}
\setlength{\headheight}{15pt}  %with titling
\setlength{\droptitle}{-1.5cm} %title 與上緣的間距
% \posttitle{\par\end{center}} % title 與內文的間距
\parindent=24pt %設定縮排的距離
% \parskip=1ex  %設定行距
% \pagestyle{empty}  % empty: 無頁碼
% \pagestyle{fancy}  % fancy: fancyhdr

% use with fancygdr
% \lhead{\leftmark}
% \chead{}
% \rhead{}
% \lfoot{}
% \cfoot{}
% \rfoot{\thepage}
% \renewcommand{\headrulewidth}{0.4pt}
% \renewcommand{\footrulewidth}{0.4pt}

% \fancypagestyle{firststyle}
% {
  % \fancyhf{}
  % \fancyfoot[C]{\footnotesize Page \thepage\ of \pageref{LastPage}}
  % \renewcommand{\headrule}{\rule{\textwidth}{\headrulewidth}}
% }
% }}}
% insert code {{{
\usepackage{listings}
\usepackage{color}

\definecolor{dkgreen}{rgb}{0,0.6,0}
\definecolor{gray}{rgb}{0.5,0.5,0.5}
\definecolor{mauve}{rgb}{0.58,0,0.82}

\lstset{
    frame=tb,
    language=Python,
    aboveskip=3mm,
    belowskip=3mm,
    showstringspaces=false,
    columns=flexible,
    basicstyle={\small\ttfamily},
    numbers=none,
    numberstyle=\tiny\color{gray},
    keywordstyle=\color{blue},
    commentstyle=\color{dkgreen},
    stringstyle=\color{mauve},
    breaklines=true,
    breakatwhitespace=true,
    tabsize=3
}
% Load from file
% \lstinputlisting[breaklines]{filename}
% }}}
% change the number of max columns in matrix {{{
\setcounter{MaxMatrixCols}{20}
% }}}
% basic {{{
% \title{}
% \author{}
% \date{\today}
% }}}
% }}}

\begin{document}
%\maketitle
{\bf \noindent
\rule[3pt]{\textwidth}{0.3pt}\\
Introduction to Artificial Intelligence
and Machine Learning \hfill \\
Instructor: Tian-Li Yu\hfill B03901133 Gary Shih \\
HW5 \hfill \today\\
\vspace{-10pt} \\
\rule[3pt]{\textwidth}{1.3pt}\\
[-1cm]
}

\section*{Problem 4}

\begin{itemize}
  \item HOG feature \\
    HOG feature is a well-known feature on classification problems (usually used by SVM).
  \item The number of separate, connected regions of white pixels
    \begin{itemize}
      \item 1 2 3 5 7
      \item 0 4 6 9
      \item 8
    \end{itemize}
  \item The number of two adjacent pixels that are not the same color in some vertical or horizontal lines
    \begin{itemize}
      \item vertical
        \begin{itemize}
          \item 1
          \item 2 3 5 6 8 9
          \item 0 4 7
        \end{itemize}
      \item horizontal
        \begin{itemize}
          \item 1 2 3 5 7
          \item 0 4 6 8 9
        \end{itemize}
    \end{itemize}
  \item The ratio of black pixels in some vertical or horizontal lines \\
    Since these three features are apparently different on some different digits (see the above approximate lists),
    they should be useful on classification problems.

\end{itemize}

\section*{Problem 6}
\begin{itemize}
  \item Distance to each food
  \item Distance to each capsule
  \item Distance to each scared ghost
  \item Distance to each ghost
\end{itemize}
The first three terms are apparently good for pacman either to win or to get
higher score, and the last term is important to pacman since pacman should not
be too close to any ghost.

\end{document}
